\documentclass{article}

\usepackage{algorithm}
\usepackage{algorithmic}
\usepackage{fullpage}
\usepackage{amsmath,amssymb,amsthm}
\newtheorem{proposition}{Proposition}
\newtheorem{lemma}{Lemma}
\newtheorem{theorem}{Theorem}
\newtheorem{definition}{Definition}
\DeclareMathOperator*{\Diag}{Diag}
\DeclareMathOperator*{\TPR}{TPR}
\DeclareMathOperator*{\Segments}{Segments}
\DeclareMathOperator*{\Changes}{Changes}
\DeclareMathOperator*{\FPR}{FPR}
\DeclareMathOperator*{\argmax}{arg\,max}
\DeclareMathOperator*{\maximize}{maximize}
\DeclareMathOperator*{\minimize}{minimize}
\newcommand{\ZZ}{\mathbb Z}
\newcommand{\NN}{\mathbb N}
\newcommand{\RR}{\mathbb R}

\begin{document}

\title{FPOP with labels}
\author{Toby Dylan Hocking}
\maketitle

We have a sequence of $d$ data points to segment,
$\mathbf z\in\mathbb R^d$. We also have $l$ labeled regions
$R=\{(\underline p_1, \overline p_1, c_1), \dots,
(\underline p_l, \overline p_l, c_l)\}$:
\begin{itemize}
\item $\underline p_i$ is the start of a labeled region,
\item $\overline p_i$ is the end of a labeled region,
\item $c_i\in\{0,1\}$ is the number of changes in the region
  $[\underline p_i, \overline p_i]$. 
\end{itemize}
For example
$R=\{(\underline p_1=1,\overline p_1=2,c_1=0),(\underline
p_2=2,\overline p_2=5,c_2=1)\}$ means that there is no change after
the first data point, and there must be exactly one change somewhere
between data points 2 and 5 (three possibilities). Assume the
labeled regions are ordered:
\begin{equation}
  \label{eq:sorted}
  1 \leq 
\underline p_1 < \overline p_1 \leq 
\underline p_2 < \overline p_2 \leq
\cdots \leq 
\underline p_l < \overline p_l \leq 
d
\end{equation}
Thus the number of possible labels is $l\in\{0, 1, \dots, d-1\}$.

\section{Optimization problem}

We would like to find the best mean vector $\mathbf m\in\mathbb R^d$.
Best is defined using the Optimal Partitioning objective function -- a
loss $\ell$ term for each data point, and a per-change penalty of
$\lambda\geq 0$ (TODO:CITATION). We also add the constraints that the
labels must be obeyed.

\begin{align}
\minimize_{
  \mathbf m\in\RR^{d}
  } &\ \ 
\lambda\sum_{t=1}^{d-1} I(m_t \neq m_{t+1})
+
\sum_{t=1}^d \ell(z_t, m_t) 
  \label{LabeledProb}
\\
    \text{subject to} 
& \ \ \sum_{t=\underline p_i}^{\overline p_i-1} I(m_t \neq m_{t+1})=c_i
\text{ for all } i\in\{1,\dots,l\}.
\end{align}
Note that $I$ is the indicator function (1 if true, 0 otherwise).

SegAnnot is one algorithm that can be used in this context, but it is
limited in that it can not detect any changes that are not labeled
(TODO:citation). In fact SegAnnot computes a model with exactly
$N_1(R)=\sum_{(\underline p_i, \overline p_i, c_i)\in R} I(c_i=1)$ changes.

The labels cover
$S(R)=\sum_{(\underline p_i, \overline p_i, c_i)\in R} \overline
p_i-\underline p_i$ possible changes, in which there are exactly
$N_1(R)$ changes. There are thus $d-S(R)$ positions which are not
labeled, which may or may not have changes. Thus the number of
possible changes is in $[N_1(R), d-S(R)+N_1(R)]$.

\section{Algorithm}

We propose a new algorithm, LabeledFPOP, which solves
(\ref{LabeledProb}). We define $C_t(\mu)$ as the optimal cost up to
data point $t$ if there is a mean of $\mu$ on the last data point
$t$. The initialization is $C_1(\mu)= \ell(z_1, \mu)$. 

\subsection{Usual FPOP}
In the usual
FPOP (without the label constraints), the dynamic programming update
rule for $t>1$ is
\begin{equation}
  \label{eq:usual_FPOP}
  C_t(\mu)=\ell(z_t, \mu) + \min
  \begin{cases}
\hat C_{t-1}+\lambda &\text{ if there is a change between $t-1$ and $t$,}\\
C_{t-1}(\mu) & \text{ if there is no change.}
  \end{cases}
\end{equation}
Note that $\hat C_{t-1}=\min_\mu C_{t-1}(\mu)$ is the optimal cost up
to data point $t-1$.

For LabeledFPOP the update rule (\ref{eq:usual_FPOP}) is used if there
is no label between data points $t-1$ and $t$. Indeed, if $R=\{\}$
then there are no labels at all, so the LabeledFPOP problem and
solution are the same as the usual FPOP.

\subsection{Negative label}
If there is a $c_i=0$ label such that
$\underline c_i < t \leq \overline c_i$, then the update rule is
simpler: the only possibility is no change.
\begin{equation}
  \label{eq:negative_update}
  C_t(\mu)=\ell(z_t,\mu)+
C_{t-1}(\mu).
\end{equation}

\subsection{Positive label at the start}
For a $c_i=1$ label the update rule is a bit more complicated. Take a
simple example to start. For a
$(\underline p_i=1,\overline p_i=3,c_i=1)$ label, the cost up to the
end of the label is
\begin{equation}
  \label{eq:positive_cost}
  \min_{\mu_1, \mu_2}
  \begin{cases}
    \ell(z_1,\mu_1) + \ell(z_2, \mu_2)+\ell(z_3, \mu_2) &\text{ for a
      change after 1}\\
    \ell(z_1,\mu_1) + \ell(z_2, \mu_1)+\ell(z_3, \mu_2) &\text{ for a
      change after 2}\\
  \end{cases}
\end{equation}
In general for a
$(\underline p_i=1,\overline p_i,c_i=1)$ label, the cost up to the end
of the label is
\begin{equation}
  \label{eq:positive_cost}
  \min_{\mu_1, \mu_2}
  \min_{t\in[1, \overline p_i-1]}
  \underbrace{
    \sum_{j=1}^t \ell(z_j, \mu_1)
  }_{\mathcal L_{1,t}(\mu_1)}+
  \underbrace{
    \sum_{j=t+1}^{\overline p_i}\ell(z_j, \mu_2)
  }_{\mathcal L_{t+1,\overline p_i}(\mu_2)}.
\end{equation}
We can use the recursive update rule for $t\in[2,\overline p_i-1]$:
\begin{equation}
  \mathcal L_{1,t}(\mu) = \mathcal L_{1,t-1}(\mu) +\ell(z_t , \mu).
\end{equation}
And then for the optimal cost we initialize $t=2$ to the cost of a
change between 1 and 2:
\begin{equation}
  C_2(\mu)=\mathcal{\hat L}_{1,1}+\lambda+\ell(z_2,\mu).
\end{equation}
And then
the update rule for
$t\in[3, \overline p_i]$ is:
\begin{equation}
  \label{eq:positive_update}
  C_t(\mu)=\ell(z_t, \mu) + \min
  \begin{cases}
\mathcal{\hat L}_{1, t-1}+\lambda &\text{ if there is a change between $t-1$ and $t$,}\\
C_{t-1}(\mu) & \text{ if the labeled change was before that.}
  \end{cases}
\end{equation}
\subsection{Positive label general case}

In general for a $(\underline p_i,\overline p_i,c_i=1)$ label, we need
to add the optimal cost function before the label, for
$t\in[\underline p_i, \overline p_i-1]$:
\begin{equation}
D_{\underline p_i, t}(\mu) = 
\begin{cases}
  C_{\underline p_i}(\mu) & \text{ for }t=\underline p_i\\
  \ell(z_t, \mu) + D_{\underline p_i,t-1}(\mu) & \text{ for }t\in[\underline p_i+1, \overline p_i-1].
\end{cases}
\end{equation}
And then for the optimal cost we initialize $t=\underline p_i+1$ to
the cost of a change between $\underline p_i$ and $\underline p_i+1$:
\begin{equation}
  C_{\underline p_i+1}(\mu)=\ell(z_{\underline p_i+1},\mu)+\hat D_{\underline p_i,\underline p_i}+\lambda.
\end{equation}
And then
the update rule for
$t\in[\underline p_i+2, \overline p_i]$ is:
\begin{equation}
  \label{eq:positive_update}
  C_t(\mu)=\ell(z_t, \mu) + \min
  \begin{cases}
\hat D_{\underline p_i, t-1}+\lambda &\text{ if there is a change between $t-1$ and $t$,}\\
C_{t-1}(\mu) & \text{ if the change was before that.}
  \end{cases}
\end{equation}

\subsection{Pseudocode}

\begin{algorithm}[H]
\begin{algorithmic}[1]
\STATE Input: data $\mathbf z\in\mathbb R^d$, 
non-negative penalty $\lambda\in\RR_+$,
sorted labels $(\underline p_i, \overline p_i, c_i)\in R$.
\STATE Output: optimal cost $\hat{\mathbf C}\in\mathbb R^d$,
mean $\mathbf m\in\mathbb R^d$,
previous segment end $\mathbf T\in\{1,\dots,d\}^d$. 
\STATE $i\gets 1$ // label counter
\STATE $\text{labelType}\gets\text{unlabeled}$
\STATE for $t$ from $1$ to $n$:
\begin{ALC@g}
  \STATE if $t==1$:
  \begin{ALC@g}
    \STATE $\text{cost}\gets 0$
  \end{ALC@g}
  \STATE else if $\text{labelType}==0$:
  \begin{ALC@g}
    \STATE $\text{cost}\gets\text{prev\_cost}$
  \end{ALC@g}
  \STATE else if $\text{labelType}==1$:
  \begin{ALC@g}
    \STATE $\text{change\_cost}\gets \text{MinPiece}(D) + \lambda$
    \STATE if $t==\underline p_i+1$:
    \begin{ALC@g}
      \STATE $\text{cost}\gets\text{change\_cost}$
    \end{ALC@g}
    \STATE else:
    \begin{ALC@g}
      \STATE $\text{cost}\gets\text{MinOfTwo}(
      \text{change\_cost}, \text{prev\_cost})$
    \end{ALC@g}
    \STATE $D$ += $\text{CostFun}(z_t)$
  \end{ALC@g}
  \STATE else if $\text{labelType}==\text{unlabeled}$:
  \begin{ALC@g}
    \STATE $\text{change\_cost}\gets\text{MinPiece}(\text{cost}) + \lambda$
    \STATE $\text{cost}\gets\text{MinOfTwo}( 
    \text{change\_cost}, \text{prev\_cost})$
  \end{ALC@g}
  \STATE $\text{cost}$ += $\text{CostFun}(z_t)$
  \STATE if $t==\overline p_i$: // what update rule should we use next?
  \begin{ALC@g}
    \STATE $\text{labelType}\gets \text{unlabeled}$
    \STATE $i\gets i+1$
  \end{ALC@g}
  \STATE if $t==\underline p_i$:
  \begin{ALC@g}
    \STATE $\text{labelType}\gets c_i$
    \STATE if $\text{labelType}==1$:
    \begin{ALC@g}
      \STATE $D\gets \text{cost}$
    \end{ALC@g}
  \end{ALC@g}
  \STATE $\hat C_t, m_t, T_t\gets \text{Minimize}(\text{cost})$
  \STATE $\text{prev\_cost}\gets\text{cost}$
\end{ALC@g}
\caption{\label{algo:LabeledFPOP}Labeled Functional Pruning Optimal
  Partitioning Algorithm.}
\end{algorithmic}
\end{algorithm}


\end{document}
